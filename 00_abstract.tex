%!TEX root = main.tex

Bayesian networks are often used to reason under uncertainty,
in order to solve tasks like risk analysis. \todo{remove?}\mytodo{I wouldn't remove this.}
\toolname is an open source tool that supports inference queries on Bayesian networks through knowledge compilation. \mytodo{references are missing}
In the knowledge compilation step, the input Bayesian network is encoded as a decision diagram. \mytodo{It's \emph{encoded} as propositional logic and \emph{compiled} to a DD} The tool supports various diagram formats, including the weighted-positive binary decision diagram which concisely encode discrete probability distributions.\todo{Not sure how much we should emphasize the WPBDD. If we do we should probably complain about numerical instability of AADDs}
Once compiled, the probabilistic knowledge base\mytodo{unexplained term 'knowledge base'} can be queried with any inference query.
To efficiently implement both steps, \toolname uses simulated annealing to split the knowledge base into a predefined number of partitions. This often further reduces the decision diagram size and crucially enables parallelism in both the compilation and the inference steps.
Experiments demonstrated that this partitioned approach, in combination with the WPBDD encoding\mytodo{the word \emph{representation} or target \emph{language} is more often used than \emph{encoding}}, can outperform other approaches\todo{which} in the knowledge compilation step, at the cost of slightly more expensive inference queries.
Additionally, the tool attains 15-fold parallel speedups using 64 cores, .





