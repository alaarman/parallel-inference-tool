\begin{minipage}[!]{\mytikzfigurewidth}
    \centering
    \begingroup
    \begin{tikzpicture}[scale=1.5*\mytikzscale,
            every path/.style={>=latex},
            inner sep=0pt,
            line width=1pt,
            thin,
            font=\mytikznormalfontsize]

        \newcommand{\myrectanglesize}{44pt}
        \newcommand{\mynodesize}{16pt}
        \newcommand{\mytikzlocalscale}{0.5}

        % bayesian network input
        \begin{scope}[shift={(0,0)}]
            \node[draw,rectangle,rounded corners,solid,thin,minimum width=\myrectanglesize,minimum height=\myrectanglesize]  (bn) at (0,0)     {};


            \begin{scope}[
                shift={(0,1)},
                scale=\mytikzlocalscale,
                every node/.style={solid,circle,draw,minimum size=\mynodesize*\mytikzlocalscale},
                every path/.style={>=latex,->},
                ]
                \node (a) at (0,0)     {};
                \node (b) at (-2,-2.5) {};
                \node (c) at ( 2,-2.5) {};
                \draw (a) edge (b);
                \draw (a) edge (c);
            \end{scope}
            \textnode (bnlabel) at (0,-1.2) {\small\verb+.net+};
        \end{scope}



        % compiler
        \begin{scope}[shift={(7,0)},every text node part/.style={align=center}]
            \node[fill=black!15,draw,rectangle,solid,thin,minimum width=70pt,minimum height=55pt]  (compiler) at (0,0)     {\compiler};

            % ordering
            \draw[->, color=black!50] (2.75, 1.8) -- (6.3, 1.8);
            \draw[->, color=black!50, dashed, rounded corners=2pt]  (6.2, 3) -- (-4,3) -- (-4, 1) -- (-2.7,1);


            % partitioning
            \draw[->, color=black!50] (2.75, -1.8) -- (5.3, -1.8);
            \draw[->, dashed, rounded corners=2pt, color=black!50]  (5.2, -3) -- (-4,-3) -- (-4, -1) -- (-2.7,-1);
            \draw[->, rounded corners=2pt, color=black!50]  (8.2, -4.7) -- (8.2,-9.8)  -- (-5.2,-9.8)-- (-5.2, -13.2) -- (-2.7, -13.2);


            % mapping
            \draw[->, color=black!50] (4.8, -1.8) -- (4.8, -5.3);
            \draw[->, rounded corners=2pt, color=black!50]  (7.2, -8.7) -- (7.2,-9.4) -- (-5.6,-9.4) -- (-5.6, -13.6) -- (-2.7, -13.6);


            % to bdds
            \draw[->, rounded corners=2pt, color=black!50] (2.75, 0.4) -- (11.8, 0.4) -- (11.8, 3) -- (13.3, 3);
            \draw[->, rounded corners=2pt, color=black!50] (2.75, 0) -- (12.3, 0);
            \draw[->, rounded corners=2pt, color=black!50] (2.75, -0.4) -- (11.8, -0.4) -- (11.8, -5.3);

            % away from bdds
            \draw[->, rounded corners=2pt, color=black!50] (16.2, 1.28) -- (16.2, -11) -- (-4,-11) -- (-4, -12)-- (-2.7, -12);
            \draw[->, rounded corners=2pt, color=black!50] (15.2, -2.73) -- (15.2, -10.6)-- (-4.4,-10.6) -- (-4.4,  -12.4) -- (-2.7, -12.4);
            \draw[->, rounded corners=2pt, color=black!50] (14.2, -8.7) -- (14.2, -10.2)-- (-4.8,-10.2) -- (-4.8, -12.8) -- (-2.7, -12.8);
            \draw[->, rounded corners=2pt, color=black!50] (-6, -1.7) -- (-6, -14)-- (-2.7, -14);


        \end{scope}

        % inference engine
        \begin{scope}[shift={(7,-14)},every text node part/.style={align=center}]
            \node[fill=black!15,draw,rectangle,solid,thin,minimum width=70pt,minimum height=55pt]  (inference) at (0,0)  
              {\splitengine};

            \node[draw,rectangle,rounded corners,solid,thin,minimum width=65pt,minimum height=\myrectanglesize]  (probs) at (8,0)     {Posterior\\probabilities};
            \draw[->, color=black!50] (inference) edge (probs);


        \end{scope}

        % ordering file
        \begin{scope}[shift={(15,3)}]
            \node[draw,rectangle,rounded corners,solid,thin,minimum width=\myrectanglesize,minimum height=\myrectanglesize]  (ordering) at (0,0)     {};

            \textnode[align=center, shift={(0,0.3)}] {Ordering\\file};
            \textnode[shift={(0,-0.5)}] {\small \verb+.ord+};
        \end{scope}

% partition file
        \begin{scope}[shift={(14,-3)}]
            \node[draw,rectangle,rounded corners,solid,thin,minimum width=\myrectanglesize,minimum height=\myrectanglesize]  (partition) at (0,0)     {};
            \textnode[align=center, shift={(0,0.3)}] {Partition\\file};
            \textnode[shift={(0,-0.5)}] {\small \verb+.part+};
        \end{scope}

        % mapping file
        \begin{scope}[shift={(13,-7)}]
            \node[draw,rectangle,rounded corners,solid,thin,minimum width=\myrectanglesize,minimum height=\myrectanglesize]  (mapping) at (0,0)     {};

            \textnode[align=center, shift={(0,0.3)}] {Mapping\\file};
            \textnode[shift={(0,-0.5)}] {\small \verb+.map+};
        \end{scope}


        % first BDD
        \begin{scope}[shift={(22,3)}]
            \node[draw,rectangle,rounded corners,solid,thin,minimum width=\myrectanglesize,minimum height=\myrectanglesize]  (bdd1) at (0,0)     {};

            \textnode[rotate=90] (bdd1label) at (1.2,0) {\small\verb+.0.bdd+};

            \begin{scope}[shift={(-0.7,1.2)},scale=\mytikzlocalscale,every node/.style={draw,minimum size=0.7*\mynodesize*\mytikzlocalscale},every path/.style={>=latex}]
                \node[circle,solid] (a) at (0,0)     {};
                \node[circle,solid] (b) at (-1,-1.5) {};
                \node[circle,solid] (c) at ( 1,-1.5) {};
                \node[circle,solid] (d) at (0,-3)     {};
                \node[circle,solid] (e) at (2,-3) {};
                \node[rectangle,solid] (true) at (0,-5) {};
                \node[rectangle,solid] (false) at (2,-5) {};

                \draw[-] (a) edge (b);
                \draw[-,densely dotted] (a) edge (c);
                \draw[-] (c) edge (d);
                \draw[-,densely dotted] (c) edge (e);
                \draw[-,bend right=20] (b) edge (true);
                \draw[-,densely dotted,bend left=20] (b) edge (false);
                \draw[-] (d) edge (true);
                \draw[-,densely dotted] (d) edge (false);
                \draw[-] (e) edge (true);
                \draw[-,densely dotted] (e) edge (false);
            \end{scope}
        \end{scope}

        % second BDD
        \begin{scope}[shift={(21,-1)}]
            \node[draw,rectangle,rounded corners,solid,thin,minimum width=\myrectanglesize,minimum height=\myrectanglesize]  (bdd2) at (0,0)     {};

            \textnode[rotate=90] (bdd2label) at (1.2,0) {\small\verb+.1.bdd+};

            \begin{scope}[xscale=-1,shift={(0.2,1.2)},scale=\mytikzlocalscale,every node/.style={draw,minimum size=0.7*\mynodesize*\mytikzlocalscale},every path/.style={>=latex}]
                \node[circle,solid] (a) at (0,0)     {};
                \node[circle,solid] (b) at (-1,-1.5) {};
                \node[circle,solid] (c) at ( 1,-1.5) {};
                \node[circle,solid] (d) at (0,-3)     {};
                \node[circle,solid] (e) at (2,-3) {};
                \node[rectangle,solid] (true) at (0,-5) {};
                \node[rectangle,solid] (false) at (2,-5) {};

                \draw[-] (a) edge (b);
                \draw[-,densely dotted] (a) edge (c);
                \draw[-] (c) edge (d);
                \draw[-,densely dotted] (c) edge (e);
                \draw[-,bend right=20] (b) edge (true);
                \draw[-,densely dotted,bend left=20] (b) edge (false);
                \draw[-] (d) edge (true);
                \draw[-,densely dotted] (d) edge (false);
                \draw[-] (e) edge (true);
                \draw[-,densely dotted] (e) edge (false);
            \end{scope}
        \end{scope}

        \begin{scope}[shift={(20.2,-4)}]
            \textnode[] (bddntag) at (0,0) {\verb+...+};

        \end{scope}


        % third BDD
        \begin{scope}[shift={(20,-7)}]
            \node[draw,rectangle,rounded corners,solid,thin,minimum width=\myrectanglesize,minimum height=\myrectanglesize]  (bdd2) at (0,0)     {};

            \textnode[rotate=90] (bddnlabel) at (1.2,0) {\small\verb+.n.bdd+};
            \textnode[] (bddntag) at (-0.5,0) {\verb+...+};
        \end{scope}


        \draw[->, color=black!50] (bn) edge (compiler);

        \begin{scope}[shift={(0,-15.2)}]

            \node[draw,rectangle,rounded corners,solid,thin,minimum width=\myrectanglesize,minimum height=25pt]  (query) at (0,0)     {Queries};
           \draw[->, color=black!50] (query) edge (4.3, 0);

        \end{scope}


\end{tikzpicture}
\endgroup
\end{minipage}
