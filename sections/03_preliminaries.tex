%!TEX root = main.tex

\section{Background}
\label{sec:background}

A Bayesian network (BN) $\mathcal{B}$ is a probabilistic graphical model that represents a joint probability distribution over its variables. Let $X = \{X_1,\ldots,X_n\}$ be a set of random variables.
%We make no distinction between singleton sets $X = \{X_1\}$ and the variable $X_1$.
Values of a variable $X_1$ are denoted in lowercase.
We denote with $P(X = x)$ the probability that $(X_1,\ldots,X_n) = (x_1,\ldots,x_n)$,
i.e. $X_i = x_i$, for $i =1,\ldots,n$.
Let  $I \subseteq [n]$, then $X_I = \{X_i \mid i \in I, X_i \in X\}$. %

\begin{definition}[Bayesian Networks]\label{def:bnfac}
    \ULforem
    %\newcommand{\powerset}{\raisebox{.15\baselineskip}{\Large\ensuremath{\wp}}}
    A \emph{Bayesian network} $\mathcal{B} = \tuple{\mathcal{G},P}$ is a DAG $\mathcal{G} = (V,E)$, with nodes $V$ and edges $E \subseteq V \times V$, that models a factorization of joint probability distribution $P(X_V)$ defined over random variables $X_V$ as:%
    \begin{equation}
    P(X_V = x_V) = \prod_{v \in V}P(X_v = x_v \mid X_{\myparents(v)}= x_{\myparents(v)}),
    \end{equation}%
    \noindent such that there is a one-to-one correspondence between nodes $V$ and variables $X_V$, and the conditional probability distribution of $X_v \in X_V$ given its parents $X_{\myparents(v)}$ is specified as $P(X_v\ |\ X_{\myparents(v)})$.

\end{definition}

\begin{example}[Bayesian Network]\label{ex:full}\label{ex:bn}
 Figure~\ref{fig:bn} shows a BN $\mathcal{B}$ defined over variables $X = \{A,B\}$ (Figure~\ref{subfig:bn}), its CPTs (Figure~\ref{subfig:cpts}) and the corresponding weighted BDD (Figure~\ref{subfig:bdd}), where atoms $\{a_1,a_2,a_3\}$ and $\{b_1,b_2\}$ map to instiations of variable $A$ and $B$, respectively.
 \vspace{-1em}
\begin{figure}[H]
    \centering

    \hfill\begin{subfigure}[b]{150pt}
        \centering
        \begin{minipage}{150pt}
            \centering
            \setlength{\tabcolsep}{4pt}
            \begin{small}
                \begin{tabular}[t]{c | c | c}
                    \normalsize{$P(A=1)$} & \normalsize{$P(A=2)$} & \normalsize{$P(A=3)$}\\\hline
                    &&\\[-2ex]
                    0.8 & 0.1 & 0.1
                \end{tabular}
            \end{small}
        \end{minipage}\\\vspace{1em}%
        \begin{minipage}{130pt}
            \centering
            \setlength{\tabcolsep}{4pt}
            \begin{small}
                \begin{tabular}[t]{l || c | c }
                    \normalsize{$A$} & \normalsize{$P(B\!=\!1 | A)$} & \normalsize{$P(B\!=\!2 | A)$}\\\hline
                    &&\\[-2ex]
                    1 &  0.5 & 0.5\\
                    2 &  0.5 & 0.5\\
                    3 &  0 & 1\\
                \end{tabular}
            \end{small}
        \end{minipage}
        \caption{Conditional probability tables.}
        \label{subfig:cpts}
    \end{subfigure}\hfill
    \begin{subfigure}[b]{110pt}
        \centering
        \begin{tikzpicture}[mytikzgraphoptions]

    % nodes
    \node (a) at (0,0)  {$\mysnl{A}$};
    \node (b) at (5,0)  {$\mysnl{B}$};
    \textnode at (3,-5)  {$P(X) = P(B|A)P(A)$};
    \textnode at (0,-7)  {};

    % edges
    \draw (a) edge (b);
\end{tikzpicture}



        \caption{Bayesian network.}
        \label{subfig:bn}
    \end{subfigure}\hfill
    \begin{subfigure}[b]{80pt}
        \centering
        \scalebox{0.9}{
        \begin{tikzpicture}[mytikzbddoptions]

    % nodes
    \node (a1) at (0,-3)     {$a_1$};
    \node (a3) at (3,-6) {$a_3$};
    \node (b1) at (0,-9)  {$b_1$};
    \node (b12) at (3,-9)  {$b_1$};
    \node (b22) at (3,-12)  {$b_2$};

    \terminalnode (false) at (3,-15) {$0$};
    \terminalnode (true) at (0,-15)   {$1$};

    % edges
    \draw[dotted] (a1) edge (a3);
    \draw[dotted,in=50,out=-65] (a3) edge (false);
    \draw[bend right=10,dotted] (b1) edge (false);
    \draw[dotted] (b12) edge (b22);
    \draw[dotted] (b22) edge (false);

    \draw  (a1)   edge node[draw=none,midway,left,xshift=-2]  {\mytikzsmallfontsize{$0.1$}}(b1);
    \draw  (a3)   edge  node[draw=none,midway,left,xshift=-3]  {\mytikzsmallfontsize{$0.8$}}(b12);
    \draw  (b1)   edge node[draw=none,midway,left,xshift=-2]  {\mytikzsmallfontsize{$0.5$}}  (true);
    \draw[bend left=33]  (b12)  edge (false);
    \draw  (b22)  edge  (true);
\end{tikzpicture}


        }

        \caption{BDD.}
        \label{subfig:bdd}
    \end{subfigure}\hfill


    \caption{Bayesian network with local structure.}
    \label{fig:bn}
\end{figure}


\end{example}

Posterior probabilities can be computed using well-known lemmas in probability theory, like Bayes’ theorem $P(X | Y) = \frac{P(X | Y)P(Y)}{P(Y)}$,
marginalization $P(X) = \sum_{y} P(X, Y = y)$, and the factorization property (Defintion~\ref{def:bnfac}).


