%!TEX root = main.tex


\section{Parallel Compilation and Inference}
\label{secjjparallel}

\begin{figure}[!t]
    \centering
    \scalebox{0.6}{
        \begin{minipage}[!]{\mytikzfigurewidth}
    \centering
    \begingroup
    \usetikzlibrary{fit}
    \tikzsetnextfilename{\mytikzfigurekey}
    \begin{tikzpicture}[scale=1.6*\mytikzscale,
            every path/.style={>=latex},
            inner sep=0pt,
            line width=1pt,
            thin,
            font=\mytikznormalfontsize]

        \newcommand{\myrectanglesize}{44pt}
        \newcommand{\mynodesize}{16pt}
        \newcommand{\mytikzlocalscale}{0.5}

        \begin{scope}[shift={(0,0)}]
            \node[draw,rectangle,rounded corners,solid,thin,minimum width=\myrectanglesize,minimum height=\myrectanglesize]  (bn) at (0,0)     {};


            \begin{scope}[
                shift={(0,0.7)},
                scale=\mytikzlocalscale,
                every node/.style={solid,circle,draw,minimum size=\mynodesize*\mytikzlocalscale},
                every path/.style={>=latex,->},
                ]
                \node (a) at (0,0)     {};
                \node (b) at (-2,-2.5) {};
                \node (c) at ( 2,-2.5) {};
                \draw (a) edge (b);
                \draw (a) edge (c);
            \end{scope}
        \end{scope}

        \begin{scope}[shift={(0,-5)}]
            \node[draw,rectangle,rounded corners,solid,thin,minimum width=\myrectanglesize,minimum height=\myrectanglesize]  (partitionedbn) at (0,0)     {};

            \begin{scope}[shift={(0,0.7)},scale=\mytikzlocalscale,every node/.style={draw,minimum size=\mynodesize*\mytikzlocalscale},every path/.style={>=latex}]
                \node[circle,solid] (a) at (0,0)     {};
                \node[circle,solid] (b) at (-2,-2.5) {};
                \node[circle,solid] (c) at ( 2,-2.5) {};
                \draw[->] (a) edge (b);
                \draw[->] (a) edge (c);

                \node[inner sep=2pt,rotate fit=52,rounded corners=2pt,draw,densely dashed,fit=(a) (b)]  {};
                \node[inner sep=2pt,rotate fit=52,rounded corners=2pt,draw,densely dashed,fit=(c)]  {};
            \end{scope}
        \end{scope}

        \begin{scope}[shift={(-5,-10)}]
            \node[draw,rectangle,rounded corners,solid,thin,minimum width=\myrectanglesize,minimum height=\myrectanglesize]  (component1) at (0,0)     {};

            \begin{scope}[shift={(0,0.7)},scale=\mytikzlocalscale,every node/.style={draw,minimum size=\mynodesize*\mytikzlocalscale},every path/.style={>=latex}]
                \node[circle,solid] (a) at (0,0)     {};
                \node[circle,solid] (b) at (-2,-2.5) {};
                \node[draw=none] (c) at ( 2,-2.5) {};
                \draw[->] (a) edge (b);

            \end{scope}
        \end{scope}

        \begin{scope}[shift={(5,-10)}]
            \node[draw,rectangle,rounded corners,solid,thin,minimum width=\myrectanglesize,minimum height=\myrectanglesize]  (component2) at (0,0)     {};

            \begin{scope}[shift={(0,0.7)},scale=\mytikzlocalscale,every node/.style={draw,minimum size=\mynodesize*\mytikzlocalscale},every path/.style={>=latex}]
                \node[draw=none] (a) at (0,0)     {};
                \node[draw=none] (b) at (-2,-2.5) {};
                \node[circle,solid] (c) at ( 2,-2.5) {};

            \end{scope}
        \end{scope}
        \begin{scope}[shift={(-5,-15)},every text node part/.style={align=center}]
            \node[draw,rectangle,rounded corners,solid,thin,minimum width=\myrectanglesize,minimum height=\myrectanglesize]  (cnf1) at (0,0)     {\scriptsize{Boolean}\\[-5pt]\scriptsize{formula}\\[-2pt] $f$};

        \end{scope}

        \begin{scope}[shift={(5,-15)},every text node part/.style={align=center}]
            \node[draw,rectangle,rounded corners,solid,thin,minimum width=\myrectanglesize,minimum height=\myrectanglesize]  (cnf2) at (0,0)     {\scriptsize{Boolean}\\[-5pt]\scriptsize{formula}\\[-2pt] $g$};
        \end{scope}


        \begin{scope}[shift={(-5,-20)}]
            \node[draw,rectangle,rounded corners,solid,thin,minimum width=\myrectanglesize,minimum height=\myrectanglesize]  (bdd1) at (0,0)     {};


            \begin{scope}[shift={(-0.2,1.2)},scale=\mytikzlocalscale,every node/.style={draw,minimum size=0.7*\mynodesize*\mytikzlocalscale},every path/.style={>=latex}]
                \node[circle,solid] (a) at (0,0)     {};
                \node[circle,solid] (b) at (-1,-1.5) {};
                \node[circle,solid] (c) at ( 1,-1.5) {};
                \node[circle,solid] (d) at (0,-3)     {};
                \node[circle,solid] (e) at (2,-3) {};
                \node[rectangle,solid] (true) at (0,-5) {};
                \node[rectangle,solid] (false) at (2,-5) {};

                \draw[-] (a) edge (b);
                \draw[-,densely dotted] (a) edge (c);
                \draw[-] (c) edge (d);
                \draw[-,densely dotted] (c) edge (e);
                \draw[-,bend right=20] (b) edge (true);
                \draw[-,densely dotted,bend left=20] (b) edge (false);
                \draw[-] (d) edge (true);
                \draw[-,densely dotted] (d) edge (false);
                \draw[-] (e) edge (true);
                \draw[-,densely dotted] (e) edge (false);
            \end{scope}
        \end{scope}


        \begin{scope}[shift={(5,-20)}]
            \node[draw,rectangle,rounded corners,solid,thin,minimum width=\myrectanglesize,minimum height=\myrectanglesize]  (bdd2) at (0,0)     {};

            \begin{scope}[xscale=-1,shift={(-0.2,1.2)},scale=\mytikzlocalscale,every node/.style={draw,minimum size=0.7*\mynodesize*\mytikzlocalscale},every path/.style={>=latex}]
                \node[circle,solid] (a) at (0,0)     {};
                \node[circle,solid] (b) at (-1,-1.5) {};
                \node[circle,solid] (c) at ( 1,-1.5) {};
                \node[circle,solid] (d) at (0,-3)     {};
                \node[circle,solid] (e) at (2,-3) {};
                \node[rectangle,solid] (true) at (0,-5) {};
                \node[rectangle,solid] (false) at (2,-5) {};

                \draw[-] (a) edge (b);
                \draw[-,densely dotted] (a) edge (c);
                \draw[-] (c) edge (d);
                \draw[-,densely dotted] (c) edge (e);
                \draw[-,bend right=20] (b) edge (true);
                \draw[-,densely dotted,bend left=20] (b) edge (false);
                \draw[-] (d) edge (true);
                \draw[-,densely dotted] (d) edge (false);
                \draw[-] (e) edge (true);
                \draw[-,densely dotted] (e) edge (false);
            \end{scope}
        \end{scope}

        \begin{scope}[shift={(0,-30)}]
            \newcommand{\mycomposedscale}{3.2}
            \node[draw,rectangle,rounded corners,solid,thin,minimum width=\mycomposedscale*\myrectanglesize,minimum height=\mycomposedscale*\myrectanglesize]  (composed) at (0,0)     {};

            \begin{scope}[shift={(0,\mycomposedscale)}]
                \node[draw,rectangle,rounded corners,densely dashed,thin,minimum width=\myrectanglesize,minimum height=\myrectanglesize]  (composed1) at (0,0)     {};

                \begin{scope}[shift={(-0.2,1.2)},scale=\mytikzlocalscale,every node/.style={draw,minimum size=0.7*\mynodesize*\mytikzlocalscale},every path/.style={>=latex}]
                    \node[circle,solid] (a) at (0,0)     {};
                    \node[circle,solid] (b) at (-1,-1.5) {};
                    \node[circle,solid] (c) at ( 1,-1.5) {};
                    \node[circle,solid] (d) at (0,-3)     {};
                    \node[circle,solid] (e) at (2,-3) {};
                    \node[rectangle,solid] (true) at (0,-5) {};
                    \node[rectangle,solid] (false) at (2,-5) {};

                    \draw[-] (a) edge (b);
                    \draw[-,densely dotted] (a) edge (c);
                    \draw[-] (c) edge (d);
                    \draw[-,densely dotted] (c) edge (e);
                    \draw[-,bend right=20] (b) edge (true);
                    \draw[-,densely dotted,bend left=20] (b) edge (false);
                    \draw[-] (d) edge (true);
                    \draw[-,densely dotted] (d) edge (false);
                    \draw[-] (e) edge (true);
                    \draw[-,densely dotted] (e) edge (false);
                \end{scope}
            \end{scope}
            \begin{scope}[shift={(\mycomposedscale,-\mycomposedscale)}]
                \node[draw,rectangle,rounded corners,densely dashed,thin,minimum width=\myrectanglesize,minimum height=\myrectanglesize]  (composed2) at (0,0)     {};

                \begin{scope}[xscale=-1,shift={(-0.2,1.2)},scale=\mytikzlocalscale,every node/.style={draw,minimum size=0.7*\mynodesize*\mytikzlocalscale},every path/.style={>=latex}]
                    \node[circle,solid] (a) at (0,0)     {};
                    \node[circle,solid] (b) at (-1,-1.5) {};
                    \node[circle,solid] (c) at ( 1,-1.5) {};
                    \node[circle,solid] (d) at (0,-3)     {};
                    \node[circle,solid] (e) at (2,-3) {};
                    \node[rectangle,solid] (true) at (0,-5) {};
                    \node[rectangle,solid] (false) at (2,-5) {};

                    \draw[-] (a) edge (b);
                    \draw[-,densely dotted] (a) edge (c);
                    \draw[-] (c) edge (d);
                    \draw[-,densely dotted] (c) edge (e);
                    \draw[-,bend right=20] (b) edge (true);
                    \draw[-,densely dotted,bend left=20] (b) edge (false);
                    \draw[-] (d) edge (true);
                    \draw[-,densely dotted] (d) edge (false);
                    \draw[-] (e) edge (true);
                    \draw[-,densely dotted] (e) edge (false);
                \end{scope}
            \end{scope}
            \begin{scope}[shift={(-\mycomposedscale,-\mycomposedscale)}]
                \node[draw,rectangle,rounded corners,densely dashed,thin,minimum width=\myrectanglesize,minimum height=\myrectanglesize]  (composed3) at (0,0)     {};

                \begin{scope}[xscale=-1,shift={(-0.2,1.2)},scale=\mytikzlocalscale,every node/.style={draw,minimum size=0.7*\mynodesize*\mytikzlocalscale},every path/.style={>=latex}]
                    \node[circle,solid] (a) at (0,0)     {};
                    \node[circle,solid] (b) at (-1,-1.5) {};
                    \node[circle,solid] (c) at ( 1,-1.5) {};
                    \node[circle,solid] (d) at (0,-3)     {};
                    \node[circle,solid] (e) at (2,-3) {};
                    \node[rectangle,solid] (true) at (0,-5) {};
                    \node[rectangle,solid] (false) at (2,-5) {};

                    \draw[-] (a) edge (b);
                    \draw[-,densely dotted] (a) edge (c);
                    \draw[-] (c) edge (d);
                    \draw[-,densely dotted] (c) edge (e);
                    \draw[-,bend right=20] (b) edge (true);
                    \draw[-,densely dotted,bend left=20] (b) edge (false);
                    \draw[-] (d) edge (true);
                    \draw[-,densely dotted] (d) edge (false);
                    \draw[-] (e) edge (true);
                    \draw[-,densely dotted] (e) edge (false);
                \end{scope}
            \end{scope}

            \draw[->,densely dashed] (composed1) edge (composed2);
            \draw[->,densely dashed] (composed1) edge (composed3);
        \end{scope}

        % edges
        \draw[->] (bn) edge (partitionedbn);
        \draw[->,shorten >= -2pt, shorten <= -2pt] (partitionedbn) edge (component1);
        \draw[->,shorten >= -2pt, shorten <= -2pt] (partitionedbn) edge (component2);
        \draw[->] (component1) edge (cnf1);
        \draw[->] (component2) edge (cnf2);
        \draw[->] (cnf1) edge (bdd1);
        \draw[->] (cnf2) edge (bdd2);
        \draw[->] (bdd1) edge (composed);
        \draw[->] (bdd2) edge (composed);

        %% independence edge
        %\begin{scope}[every path/.style={
        %        >=stealth,
        %        decoration = {snake,pre length=7pt,post length=9pt},
        %        decorate,
        %        densely dashed,
        %        shorten >= 4pt, shorten <= 4pt}]

        %        \usetikzlibrary{arrows, decorations.pathmorphing}

        %        \draw[<->] (component1) -- (component2);
        %        \draw[<->] (cnf1) -- (cnf2);
        %        \draw[<->] (bdd1) -- (bdd2);
        %        \draw[<->,shorten >= 0.5pt, shorten <= 0.5pt,decoration = {snake,pre length=3pt,post length=2pt}] (composed2) -- (composed3);
        %        \begin{scope}[shift={(1.5,-0.5)}]
        %            \draw[<->,decoration = {snake,pre length=7pt,post length=6pt},] (7,-33.5) -- node[rectangle,inner sep=0pt,draw=none,midway,yshift=-\myfontsize] {Independencies} (14,-33.5);
        %        \end{scope}

        %\end{scope}
        %\node[draw,rectangle,rounded corners,densely dotted,thin,minimum width=60pt,minimum height=30pt] (x) at (12,-34.7)     {};

        \begin{scope}[every node/.style={
                rotate=45,
                draw=none,
                anchor=west,
                align=left}]
            %\draw[-,dotted,shorten >= 10pt, shorten <= 10pt] (bn) -- (8,0);
            %\draw[-,dotted,shorten >= 10pt, shorten <= 10pt] (partitionedbn) -- (8,-5);
            \usetikzlibrary{calc}
            \textnode (bnlabel) at (10,0) {Bayesian\\Network};
            \draw[-,densely dotted,shorten >= 5pt, shorten <= 5pt] (bn) -- (bnlabel.west);

            \textnode (partitionlabel) at (10,-5) {Partitioning};
            \draw[-,densely dotted,shorten >= 5pt, shorten <= 5pt] (partitionedbn) -- (partitionlabel.west);
            \draw[-,rounded corners,densely dotted,shorten <= 5pt] (component2) -- ($(component2)+(3.4,0)$) --  ($(partitionedbn)+(8.4,0)$);

            \textnode  (encodinglabel) at (10,-15) {Encoding};
            \draw[-,densely dotted,shorten >= 5pt, shorten <= 5pt] (cnf2) -- (encodinglabel.west);

            \textnode  (compilationlabel) at (10,-20) {Decision\\ Diagrams};
            \draw[-,densely dotted,shorten >= 5pt, shorten <= 5pt] (bdd2) -- (compilationlabel.west);

            \textnode  (compositionlabel) at (10,-30) {Composed\\Representation};
            \draw[-,densely dotted,shorten >= 5pt, shorten <= 5pt] (composed) -- (compositionlabel.west);
        \end{scope}

        \draw[-,densely dotted] (-7,2.5) -- (-8,2.5) -- (-8,-22.5) node[rectangle,inner sep=0pt,draw=none,rotate=90,midway,yshift=8pt] {Compilation} -- (-7,-22.5);
        \draw[-,densely dotted] (-7,-23.5) -- (-8,-23.5)  -- (-8,-36.5) node[rectangle,inner sep=0pt,draw=none,rotate=90,midway,yshift=8pt] {Inference} -- (-7,-36.5);
\end{tikzpicture}
\endgroup
\end{minipage}

    }
    \caption{The compositional framework.}
    \label{fig:frameworkoverview}
\end{figure}

\begin{figure}[!t]
    \centering
    \begin{minipage}[!]{\mytikzfigurewidth}
    \centering
    \begingroup
    \begin{tikzpicture}[scale=1.5*\mytikzscale,
            every path/.style={>=latex},
            inner sep=0pt,
            line width=1pt,
            thin,
            font=\mytikznormalfontsize]

        \newcommand{\myrectanglesize}{44pt}
        \newcommand{\mynodesize}{16pt}
        \newcommand{\mytikzlocalscale}{0.5}

        % bayesian network input
        \begin{scope}[shift={(0,0)}]
            \node[draw,rectangle,rounded corners,solid,thin,minimum width=\myrectanglesize,minimum height=\myrectanglesize]  (bn) at (0,0)     {};


            \begin{scope}[
                shift={(0,1)},
                scale=\mytikzlocalscale,
                every node/.style={solid,circle,draw,minimum size=\mynodesize*\mytikzlocalscale},
                every path/.style={>=latex,->},
                ]
                \node (a) at (0,0)     {};
                \node (b) at (-2,-2.5) {};
                \node (c) at ( 2,-2.5) {};
                \draw (a) edge (b);
                \draw (a) edge (c);
            \end{scope}
            \textnode (bnlabel) at (0,-1.2) {\small\verb+.net+};
        \end{scope}


        % compiler
        \begin{scope}[shift={(7,0)},every text node part/.style={align=center}]
            \node[draw,rectangle,solid,thin,minimum width=70pt,minimum height=55pt]  (compiler) at (0,0)     {\verb+COMPILER+};

        \end{scope}

        % inference engine
        \begin{scope}[shift={(7,-12)},every text node part/.style={align=center}]
            \node[draw,rectangle,solid,thin,minimum width=70pt,minimum height=55pt]  (inference) at (0,0)     {\verb+INFERENCE+\\\verb+ENGINE+};

        \end{scope}

        % ordering file
        \begin{scope}[shift={(15,8)}]
            \node[draw,rectangle,rounded corners,solid,thin,minimum width=\myrectanglesize,minimum height=\myrectanglesize]  (ordering) at (0,0)     {};

            \textnode[align=center, shift={(0,0.3)}] {Ordering\\file};
            \textnode[shift={(0,-0.5)}] {\small \verb+.ord+};
        \end{scope}

% partition file
        \begin{scope}[shift={(14,3)}]
            \node[draw,rectangle,rounded corners,solid,thin,minimum width=\myrectanglesize,minimum height=\myrectanglesize]  (partition) at (0,0)     {};
            \textnode[align=center, shift={(0,0.3)}] {Partition\\file};
            \textnode[shift={(0,-0.5)}] {\small \verb+.part+};
        \end{scope}

        % mapping file
        \begin{scope}[shift={(14,-3)}]
            \node[draw,rectangle,rounded corners,solid,thin,minimum width=\myrectanglesize,minimum height=\myrectanglesize]  (mapping) at (0,0)     {};

            \textnode[align=center, shift={(0,0.3)}] {Mapping\\file};
            \textnode[shift={(0,-0.5)}] {\small \verb+.map+};
        \end{scope}


        % first BDD
        \begin{scope}[shift={(23,8)}]
            \node[draw,rectangle,rounded corners,solid,thin,minimum width=\myrectanglesize,minimum height=\myrectanglesize]  (bdd1) at (0,0)     {};

            \textnode[rotate=90] (bdd1label) at (1.2,0) {\small\verb+.0.bdd+};

            \begin{scope}[shift={(-0.7,1.2)},scale=\mytikzlocalscale,every node/.style={draw,minimum size=0.7*\mynodesize*\mytikzlocalscale},every path/.style={>=latex}]
                \node[circle,solid] (a) at (0,0)     {};
                \node[circle,solid] (b) at (-1,-1.5) {};
                \node[circle,solid] (c) at ( 1,-1.5) {};
                \node[circle,solid] (d) at (0,-3)     {};
                \node[circle,solid] (e) at (2,-3) {};
                \node[rectangle,solid] (true) at (0,-5) {};
                \node[rectangle,solid] (false) at (2,-5) {};

                \draw[-] (a) edge (b);
                \draw[-,densely dotted] (a) edge (c);
                \draw[-] (c) edge (d);
                \draw[-,densely dotted] (c) edge (e);
                \draw[-,bend right=20] (b) edge (true);
                \draw[-,densely dotted,bend left=20] (b) edge (false);
                \draw[-] (d) edge (true);
                \draw[-,densely dotted] (d) edge (false);
                \draw[-] (e) edge (true);
                \draw[-,densely dotted] (e) edge (false);
            \end{scope}
        \end{scope}

        % second BDD
        \begin{scope}[shift={(22,3)}]
            \node[draw,rectangle,rounded corners,solid,thin,minimum width=\myrectanglesize,minimum height=\myrectanglesize]  (bdd2) at (0,0)     {};

            \textnode[rotate=90] (bdd2label) at (1.2,0) {\small\verb+.1.bdd+};

            \begin{scope}[xscale=-1,shift={(0.2,1.2)},scale=\mytikzlocalscale,every node/.style={draw,minimum size=0.7*\mynodesize*\mytikzlocalscale},every path/.style={>=latex}]
                \node[circle,solid] (a) at (0,0)     {};
                \node[circle,solid] (b) at (-1,-1.5) {};
                \node[circle,solid] (c) at ( 1,-1.5) {};
                \node[circle,solid] (d) at (0,-3)     {};
                \node[circle,solid] (e) at (2,-3) {};
                \node[rectangle,solid] (true) at (0,-5) {};
                \node[rectangle,solid] (false) at (2,-5) {};

                \draw[-] (a) edge (b);
                \draw[-,densely dotted] (a) edge (c);
                \draw[-] (c) edge (d);
                \draw[-,densely dotted] (c) edge (e);
                \draw[-,bend right=20] (b) edge (true);
                \draw[-,densely dotted,bend left=20] (b) edge (false);
                \draw[-] (d) edge (true);
                \draw[-,densely dotted] (d) edge (false);
                \draw[-] (e) edge (true);
                \draw[-,densely dotted] (e) edge (false);
            \end{scope}
        \end{scope}

        % third BDD
        \begin{scope}[shift={(21,-2)}]
            \node[draw,rectangle,rounded corners,solid,thin,minimum width=\myrectanglesize,minimum height=\myrectanglesize]  (bdd2) at (0,0)     {};

            \textnode[rotate=90] (bddnlabel) at (1.2,0) {\small\verb+.n.bdd+};
            \textnode[] (bddntag) at (-0.5,0) {\verb+...+};


        \end{scope}


        \begin{scope}[every node/.style={
                rotate=45,
                draw=none,
                anchor=west,
                align=left}]
            %\draw[-,dotted,shorten >= 10pt, shorten <= 10pt] (bn) -- (8,0);
            %\draw[-,dotted,shorten >= 10pt, shorten <= 10pt] (partitionedbn) -- (8,-5);
            \usetikzlibrary{calc}


        \end{scope}

\end{tikzpicture}
\endgroup
\end{minipage}

    \caption{The implementation.}
    \label{fig:implementation}
\end{figure}

Bayesian networks represent concise factorizations of a probability distribution by using conditional independence assumptions. A more expressive model must be used to further improve a BN's factorization in order to exploit additional independences~\cite{zhang1996exploiting,boutilier1996context,friedman1998learning}.

\toolname implements inference through \emph{weighted model counting} (WMC). The goal of inference by WMC is reducing size and reasoning requirements by exploiting unused independencies, in order to perform inference more efficiently. We encode probability distributions as Boolean functions into a succinct representation language.
% Let's back up for a moment, by describing the means to this end.
AS BNs are defined over multi-valued domains, to encode the probability distribution it represents, we require an encoding is to transition from the multi-valued domain to the Boolean domain. There are multiple ways to do this, but \toolname chooses to first translate a BN into a satisfiability (SAT) instance in Conjunctive Normal Form (CNF) with dedicated variables to represent probabilities~\cite{dal2017wpbdd} (in this step we do not need to introduce extra variables as e.g. a Tseitin transformation would).

The SAT instance serves as an entry point into for the compilers to create a binary decision diagram (BDD), for instance. Compiling a BN into a BDD-like representations is commonly referred to as \emph{knowledge compilation}~\cite{darwiche2002knowledge}, or simply compilation. The \toolname compiler however specifically targets \emph{Weighted Positive Binary Decision Diagrams} (WPBDD), which is a dedicated representation for this purpose. (We discuss differences with other representations in Section~\ref{sec:conclusion}.)
	In addition, the compiler introduces a partitioning to further improve overall performance.

Figure~\ref{fig:frameworkoverview} shows a high level overview of the tool's internal processes. \toolname implements WMC in two major parts: \emph{compilation} and \emph{inference}. Let's dive into the compilation part. Given a user provided number of partitions, a partitioning is found for the BN. This partitioning is optimized by minimizing the sum of each partitions tree-width using \emph{simulated annealing}. Tree-width is a metric commonly use to indicate the complexity of BNs~\cite{bollig2014width}.

With the partitioning in hand, the following steps can be performed in parallel, per partition. Theoretically, compilation is as fast as the slowest compiling partition~\cite{dal2018parallel}. Each partition is considered an independent BN from this point on. The BN is encoded as a Boolean formula as previously mentioned~\cite{chavira2008probabilistic}. \toolname takes this Boolean formula and compiles it to a WPBDD. The most expensive operation amongst them all. Performance is primarily determined by this step. We have now reached the end of the compilation part as indicated by Figure~\ref{fig:frameworkoverview}, yielding a WPBDD per partition.

We arrive at the inference part of \toolname. The upside of getting this far, is the computational complexity of inference is linear in the size of the target representation~\cite{darwiche2002knowledge}, in our case a WPBDD. Inference is performed by traversing the target representation whilst evaluating the underlying arithmetic formula. The arithmetic formula is different for every target representation, but generally we can convert a logical OR to addition, logical AND to multiplication, and substitute variables with the value or probability they represent. All that is left, is to evaluate this formula in order to obtain the marginal probability we seek.

In case we chose to partition the BN, we compose the obtained WPBDDs and create a monolithic WPBDD. A partition is connected to another of they share a common variable. This implies that order in which we traverse partitions is not a total ordering. It is partial, and can be represented by a tree.The order in which we choose to traverse partitions determines how they are connected. As we traverse one partition, its sink is connected to the next partitions root.

