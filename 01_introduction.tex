%!TEX root = main.tex

\section{Introduction}
\label{sec:introduction}


\todo[inline]{
A motivation as to why the tool is interesting and significant should be provided. Further, the paper should describe aspects such as, for example, the assumptions about application domain and/or extent of potential generality, demonstrate the tool workflow(s), explain integration and/or human interaction, evaluate the overall role and the impact to the development process.
}


Pro's

- uses a fully symbolic (WPBDD) encoding, avoiding numerical instability in real-based DDs

- can handle any network?

- Does parallel inference (unique?)

- The compositional approach through partitioning offers various advantages. In the first place, it can reduce the effort spent on compilation because the decomposition of a propositional theory is known to yield smaller symbolic representations, which has previously been shown in model checking~\cite{narayan1996partitioned,sahoo2004partitioning,grumberg2006work} and is confirmed by our experiments (which show orders of magnitudes improvement in Table X.


Con's 

- This benefit of partitioning in compilation is offset by a potential increase in the time spent on inference, although our experiments still demonstrate good performance due to the smaller representations.

-  ?
